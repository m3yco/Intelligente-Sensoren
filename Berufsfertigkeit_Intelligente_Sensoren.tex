\documentclass[a4paper,12pt]{scrartcl}
\usepackage[ngerman]{babel}
\usepackage[T1]{fontenc}
\usepackage{lmodern}
\usepackage{blindtext}
\usepackage{tabularx}
\usepackage[utf8]{inputenc}
\usepackage{amsmath}
\usepackage{tikz}
\usetikzlibrary{arrows,shapes,positioning,shadows,trees}
\usepackage{forest}
\usetikzlibrary{shadows,arrows.meta}
\usepackage{rotating}
\usepackage{geometry}
\usepackage{graphicx}
\usepackage{acronym}
\usepackage{amsfonts}
\usepackage{hhline,booktabs}
\usepackage{siunitx}
\usepackage{amssymb}% http://ctan.org/pkg/amssymb
\usepackage{pifont}% http://ctan.org/pkg/pifont
\usepackage{textcomp}
\usepackage{eurosym}
\usepackage{forest}
\usepackage{tikz-qtree}
\usetikzlibrary{arrows.meta, shapes.geometric, calc, shadows}
\usepackage{booktabs}
\usepackage{dcolumn}
\makeatletter
\newcolumntype{d}[1]{>{\DC@{,}{,}{#1}}l<{\DC@end}}
\makeatother
\usetikzlibrary{positioning}
\usepackage{mathtools, nccmath}
\usepackage{adjustbox}
\usepackage{pgfgantt}
\usetikzlibrary{calc} 
\usetikzlibrary{arrows.meta} 
\usetikzlibrary{positioning}
\usepackage{verbatim}
\usetikzlibrary{arrows.meta}
\usepackage{booktabs}
\usepackage{xcolor}
\usepackage{colortbl}
\usepackage{longtable}
\usepackage{ulem}
\usepackage{amsthm}
\usepackage{ulem}
\usepackage{setspace}

%Zitationsstil
\bibliographystyle{unsrt}
%cite{...} -> Drag and Drop von Citavi

%gantt
\newganttchartelement*{rresource}{
    rresource/.style={},
    inline=true,
    rresource inline label node/.style={anchor=west,font=\bfseries\itshape\color{blue}},
    rresource left shift=0ex,
    rresource right shift=0ex
}
\newganttchartelement*{lresource}{ % The starred version mimics a milestone element with 2 options
    lresource/.style={}, % Don't draw the node
    inline=true,
    lresource inline label node/.style={anchor=east,font=\bfseries\itshape\color{blue}},
    lresource left shift=0ex,
    lresource right shift=0ex
}
%gantt
\makeatletter
\newcommand{\ccell}[3][]{%
  \kern-\fboxsep
  \if\relax\detokenize{#1}\relax
    \expandafter\@firstoftwo
  \else
    \expandafter\@secondoftwo
  \fi
  {\colorbox{#2}}%
  {\colorbox[#1]{#2}}%
  {#3}\kern-\fboxsep
}
\makeatother
\definecolor{cellgray}{gray}{0.9}
\definecolor{pastelred}{rgb}{1.0, 0.41, 0.38}
\definecolor{celadon}{rgb}{0.67, 0.88, 0.69}
\definecolor{corn}{rgb}{0.98, 0.93, 0.36}

\newcolumntype{x}{>{\columncolor{celadon}}c}
\newcolumntype{y}{>{\columncolor{corn}}c}
\newcolumntype{z}{>{\columncolor{pastelred}}c}

\DeclarePairedDelimiter{\nint}\lfloor\rceil
\usepackage{varwidth}
\newcommand\Umbruch[2][3cm]{\begin{varwidth}{#1}\centering#2\end{varwidth}}
\newcommand\Zelle[2][2cm]{\begin{varwidth}{#1}\flushleft#2\end{varwidth}}
\newcommand\Absatz[2][12cm]{\begin{varwidth}{#1}\flushleft#2\end{varwidth}}
\newcommand\Kommentar[2][9.5cm]{\begin{varwidth}{#1}\flushleft#2\end{varwidth}}
\newcommand\Risiko[2][2.5cm]{\begin{varwidth}{#1}\flushleft#2\end{varwidth}}

\tikzset{
  basic/.style  = {draw, text width=2cm, drop shadow, font=\sffamily, rectangle},
  root/.style   = {basic, rounded corners=2pt, thin, align=center,
                   fill=red!30},
  level 2/.style = {basic, rounded corners=4pt, thin,align=center, fill=gray!30,
                   text width=9em},
  level 3/.style = {basic, thin, align=left, fill=green!30, text width=8em}
}
\textwidth158mm
\begin{document}

\title{Studienarbeit \vspace{20px} \hfill \\ Berufsfertigkeit\\ Wintersemester 18/19 \vspace{20px} \hfill \\  \vspace{50px} 
Intelligente Sensoren - SMARTe Zukunft \hfill \\ \hfill \\
\hfill \\ 
\begin{center}
\includegraphics[width=10cm]{picture/hs_albsig_logo}
\end{center}
\hfill \\  \vspace{10px}
}


<<<<<<< HEAD
\author{Martin Hafner (ITS) 86114 \hfill \\
Maik Dürr (TI) 86215 \hfill \\
Domenico Milazzo (TI) 86217 \hfill \\
Ömer Ozer (TI) 85115 \hfill \\
}
\vspace{10px}
\date{\textbf{15.02.2019}}
=======
\author{Martin Hafner Matrikelnummer 86114 \hfill \\
Maik Dürr Matrikelnummer 86215 \hfill \\
Domenico Milazzo Matrikelnummer 86217 \hfill \\
Ömer Ozer Matrikelnummer 85115 \hfill \\}
\date{15.02.2019}
>>>>>>> 27f486b137010713ada8ec164ee9d0fe381e6ad8
\maketitle
\thispagestyle{empty}
\clearpage
\tableofcontents
%\thispagestyle{empty}
\listoffigures
\listoftables
\paragraph{\Large{Abkürzungsverzeichnis}}
\begin{acronym}[Bash]
 \acro{IS}{Intelligente Sensoren}
\end{acronym}
\onehalfspacing

\paragraph{\large Abstract}
\begin{abstract}
Die Welt bewegt sich immer mehr in Richtung Intelligente Systeme und immer mehr auf Verarbeitung von großen Datenströmen. Um diese fülle an Anforderungen meistern zu können braucht es Werkzeuge die diese Datenströme erfassen und verarbeiten können. Diese Werkzeuge sollen nicht nur zur Bewältigung von Problemen helfen, sondern sollen diese auch noch auswerten. Dafür benötigen die Maschinen und Geräte der Zukunft Organe wie Lebewesen. Eine neue Art von Technologie muss sich dafür etablieren, diese werden Intelligente Sensoren oder auch Smart Sensors genannt.
Intelligente Sensoren werden als Schlüsseltechnologie der Industrie 4.0 und für eine Smarte Zukunft bezeichnet. Anwendungsbeispielen zeigen und die Pioniere in diesem neuem Zeitalter und gestalten mit Visionen eine neue Zukunft.
\end{abstract}

<<<<<<< HEAD
\section{Einleitung}
=======
\section{Grundlagen}

\section{Einsatzgebiete für Intelligente Sensoren}

\section{Zukunftsaussichten}

\section{Schlusswort}
>>>>>>> 27f486b137010713ada8ec164ee9d0fe381e6ad8

\section{Grundlagen}


\section{Einsatzgebiete für Intelligente Sensoren}

\section{Zukunftsaussichten}

\section{Schlusswort}

\setcounter{page}{1}

\newpage



\clearpage
\bibliography{LiteraturInteligenteSeneoren}

\end{document}